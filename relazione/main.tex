% !TEX encoding = UTF-8
% !TEX TS-program = pdflatex
% !TEX root = ../main.tex
% !TEX spellcheck = it-IT

\documentclass[final, 11pt, a4paper, titlepage]{article}
\makeatletter
\AtBeginDocument{\let\hl\@firstofone}
\makeatother

\usepackage[italian]{babel}
\usepackage[utf8]{inputenc}
\usepackage[hidelinks]{hyperref}
\usepackage{graphicx}
\usepackage{subcaption}
\usepackage{textcomp}
\usepackage{wallpaper}
\usepackage{color}
\usepackage{mathtools}
\usepackage{amssymb}
\usepackage{listings}
\usepackage[margin=1in]{geometry}
\usepackage[
	backend=biber,
	citestyle=numeric-comp,
	hyperref,
	backref,
	sorting=none
]{biblatex}

\graphicspath{{images/}}

\addbibresource{bibliography.bib}


\defbibheading{bibliography}
{
    \phantomsection 
    \addcontentsline{toc}{section}{\bibname}
    \section*{\bibname\markboth{\bibname}{\bibname}}
}

\setlength\bibitemsep{1.5\itemsep} 

%\DeclareBibliographyCategory{sampleCategory}

%\addtocategory{sampleCategory}{referenceID}


\newcommand{\university}{Università degli Studi di Padova}
\newcommand{\dept}{Dipartimento di Matematica}
\newcommand{\faculty}{Laurea Magistrale in Informatica}
\newcommand{\myyear}{Anno Accademico 2016-17}
\renewcommand{\title}{Making Android run on time}
\newcommand{\subtitle}{Relazione}
\renewcommand{\author}{Matteo Di Pirro}
\newcommand{\matr}{1154231}

\begin{document}
	\input{config/titlepage}
	\tableofcontents
	\newpage
	\section{Java per sistemi real-time}
L'uso di Java per sistemi real-time non è diffuso per varie ragioni. Le applicazioni Java vengono eseguite su una JVM in un sistema operativo general-purpose che può solo sperare di soddisfare requisiti di response time nell'ordine delle centinaia di millisecondi. Molti aspetti diversi del linguaggio sono responsabili di questi ritardi: la gestione dei thread, il caricamento dinamico delle classi, la compilazione Just-in-Time (JIT) e la garbage collection (GC). Alcune di questi effetti possono essere mitigati in fase di progettazione, ma solo con moltissimi sforzi. 

\subsection{Gestione dei thread}
Java non dà nessuna garanzia sullo scheduling o sull'utilizzo di priorità. Un'applicazione che deve rispondere agli eventi in un tempo ben definito non ha nessun modo di assicurare che un thread a bassa priorità non venga eseguito al posto di uno con priorità più alta. Per compensare, un programmatore dovrebbe dividere l'applicazione in ''sotto-applicazioni'' e farle eseguire a diverse priorità. Questo partizionamento, tuttavia, comporta una maggiore difficoltà di comunicazione e un overhead aggiuntivo.

\subsection{Caricamento dinamico delle classi}
Una JVM che rispetti la specifica di Java deve ritardare il caricamento delle classi fino a quando queste non vengono per la prima volta riferite nel programma. Il caricamento, quando avviene, può richiedere una quantità variabile di tempo, dipendentemente dal supporto fisico (disco o altro) dal quale la classe viene caricata e dalla dimensione. Generalmente il ritardo introdotto è oltre i 10ms. Di conseguenza, se decine o centinaia di classi vengono caricate, il caricamento può portare ad un ritardi significativo. Un design attento può caricare tutte le classi allo start-up, ma questa procedura va fatta manualmente.

\subsection{Garbage Collection}
La garbage collection ha vari benefici rispetto ad applicazioni classiche: pointer safety, leak avoidance e libera i programmatori dal dover gestire la memoria manualmente (tedioso e molto error-prone). Sfortunatamente però la garbage collection viene avviata automanticamente quando lo heap viene esaurito al punto tale che una richiesta di allocazione non può essere esaudita. Anche l'applicazione può chiedere una collection.

La pulizia della memoria avviene tramite la cosiddetta politica \textbf{Stopping the World}. L'applicazione viene messa in pausa per permettere al GC di pulire la memoria. Gli oggetti vivi vengono tracciati a partire da un insieme radice (oggetti puntati da campi statici, oggetti vivi nello stack dei thread, ecc). La memoria che non contiene questi oggetti viene libreata per una futura allocazione. Le pause introdotte dal GC sono illimitate in lunghezza e tipicamente sono molto intrusive (range centinaia di ms a secondi). La durata dipende dalla dimensione dello heap, dal numero di oggetti vivi e dal grado di aggressività del GC. Algoritmi più moderni utilizzano tecniche concorrenti o incrementali per ridurre il tempo di pausa, ma anche con queste tecniche non esiste un limite fissato alle attività di pulizia.

Se da un lato il non doversi preoccupare della gestione della memoria è un grande vantaggio, dall'altro la GC può introdurre pesanti ritardi, impossibili da prevedere staticamente. L'unica soluzione al problema è non usare affatto la GC, ma non è una soluzione praticabile sia per la complessità del codice da gestire sia perché difficilmente si possono trovare librerie esterne che non la utilizzano.

\subsection{Compilazione}
La maggior parte delle JVM commerciali compilano in codice nativo le parti dell'applicazione utilizzate più di frequente. Sfortunatamente l'esecuzione di codice compilato o interpretato avviene con tempi anche molto diversi tra loro. Per un'applicazione real-time l'impossibilità di prevedere quando il codice Java verrà compilato in codice nativo introduce troppo non determinismo per poter analizzare i tempi di esecuzione. Una soluzione è compilare a mano una lista di metodi che si sa essere eseguiti di frequente, ma l'operazione è molto error-prone.	
	\begin{frame}{Real-Time specification for Java \\e altre soluzioni}
	\begin{itemize}
		\item Scheduling
		\begin{itemize}
			\item Utilizzo reale delle priorità
			\item Basic Priority Inheritance Protocol
			\item Ceiling Priority Protocol (opzionale)
		\end{itemize}
		\item Gestione della memoria
		\begin{itemize}
			\item Scoped
			\item Immortal
		\end{itemize}
		\item Compilazione Ahead of time
		\begin{itemize}
			\item Maggiore prevedibilità
		\end{itemize}
	\end{itemize}
\end{frame}
	\section{Confronto delle tecniche di compilazione}
Storicamente le performance delle applicazioni Java sono sempre state molto criticate. Java è stato progettato per essere interpretato e portabile: i primi runtime avevano performance significativamente più basse dei concorrenti. Nel corso degli ultimi anni, però, i runtime Java hanno introdotto compilatori dinamici molto sofisticati: i compilatori JIT. Questi compilano selettivamente i metodi più frequentemente eseguiti in codice nativo durante l'esecuzione. Il fatto di compilare i metodi durante l'esecuzione, e non allo start-up, mantiene la portabilità. Grazie a questo, inoltre, le performance risultano incredibilmente migliorate. 

\subsection{Compilazione Just-in-time}
Figura\ref{fig:jit} mostra un esempio di compilazione just-in-time. 
\begin{figure}
	\centering
	\includegraphics[width=0.7\linewidth]{jvmjit}
	\caption[Compilatore JIT]{Compilatore JIT}
	\label{fig:jit}
\end{figure}
I programmi Java vengono compilati un metodo alla volta mentre eseguono, per raggiungere performance migliori. Durante il processo viene generata una rappresentazione interna del metodo (differente dal bytecode, ma ad un livello più alto delle istruzioni macchina). Il compilatore poi esegue una serie di ottimizzazioni per migliorare la qualità e l'efficienza del codice finale e infine traduce tutto in istruzioni macchina del processore in uso. Il compilatore opera in un thread differente cosicché l'applicazione non è costretta a bloccarsi in attesa della fine della compilazione. Un framework (\texttt{Profile}) osserva il comportamento del programma per identificare i metodi eseguiti più frequentemente. 

L'eseguire la compilazione concorrentemente all'esecuzione mantiene l'indipendenza dalla piattaforma, ma ad un costo: il tempo richiesto per compilare si somma al tempo richiesto per l'esecuzione. Ci sono due soluzioni al problema:
\begin{itemize}
	\item compilare tutto il codice senza però effettuare analisi o ottimizzazioni costose (per mantenere veloce il processo). L'overhead in questo caso è talmente piccolo che è completamente recuperato dall'incremento di performance; 
	\item compilare solo metodi eseguiti veramente frequentemente (metodi \textit{hot}). L'overhead è mantenuto basso perché molte applicazioni eseguono frequentemente solamente una piccola parte del codice, quindi è sufficiente compilare quello per ottenere un significativo aumento di performance.
\end{itemize}

\subsubsection{Vantaggi}
I compilatori JIT possono analizzare l'esecuzione e identificare le situazioni che si avverano più comunemente. A partire da queste possono poi compilare il codice (anche con ottimizzazioni molto spinte) per raggiungere performance altissime. Un esempio è dato da una semplice procedura di copia di un array, \texttt{arrayCopy}. Se viene rilevato che la dimensione dell'array è praticamente costante allora è possibile generare codice ottimizzato per quella lunghezza. 

\subsubsection{Svantaggi}
Dato che è necessaria una fase di ''training'' per capire quali parti compilare, spesso le migliori performance si ottengono dopo un po'. Inoltre i metodi eseguiti frequentemente in questo periodo iniziale potrebbero non essere effettivamente quelli più significativi. Ad ogni modo le tecniche di analisi utilizzate oggigiorno eliminano il problema. 

Alcune applicazioni, però, non possono tollerare il ritardo introdotto dalla compilazione. Un esempio sono quelle da eseguire real-time.

\subsection{Compilazione Ahead-of-time}
La compilazione AOT la trasformazione in codice nativo avviene a priori, prima dell'esecuzione. Questo evita i problemi di analisi del codice e del ritardo di esecuzione, ma introduce altre problematiche. Una di queste è dovuta al caricamento dinamico delle classi. Il compilatore in questo caso non può fare nessuna assunzione riguardo a quali classi saranno caricate. Queste possono risiedere in altre macchine oppure non esistere affatto prima dell'esecuzione (con meccanismi di reflection è possibile creare classi ''al volo'', durante l'esecuzione). Questa è una grande limitazione, perché inibisce alcune delle più importanti ottimizzazioni effettuate dai compilatori, tra cui l'inlining. 

Il codice deve quindi essere generato con tutti i riferimenti non risolti. Durante l'esecuzione ogni riferimento utilizzato è aggiornato con il suo valore reale. Questo può comportare una penalità nella prima esecuzione, perché tutti i riferimenti sono sconosciuti, ma le esecuzioni successive non soffriranno di ritardi.

Compilare tutto il codice può non essere una buona scelta: il codice nativo occupa generalmente più spazio, e molti metodi sono utilizzati così raramente che la loro compilazione non porta benefici. Tuttavia invocare metodi interpretati da metodi compilati (o viceversa) richiede molto più tempo che invocare metodi interpretati da metodi interpretati. Un compilatore JIT può risolvere il problema al volo, ma uno AOT deve selezionare attentamente cosa compilare e cosa no, a priori. 

\subsubsection{Vantaggi}
Il codice AOT, sebbene più lento di quello JIT, è molto più veloce del codice interpretato. Inoltre l'incremento di performance si ottiene più velocemente, perché non si devono compilare al volo metodi eseguiti frequentemente. Le applicazioni real-time, in particolare, traggono numerosi benefici da questo approccio. Le performance sono migliori e più deterministiche rispetto a JIT. 

\subsection{Confronto}
Figura\ref{fig:performanceaotvsjit} mostra un confronto basato sulle performance. 
\begin{figure}[h]
	\centering
	\includegraphics[width=0.7\linewidth]{performanceaotvsjit}
	\caption[Confronto di performance]{Confronto di performance}
	\label{fig:performanceaotvsjit}
\end{figure}

Inizialmente JIT è molto peggiore, perché tutti i metodi sono interpretati. Man mano che vengono compilati però le prestazioni migliorano fino a superare AOT e a raggiungere un picco. AOT, d'altra parte, ha un inizio migliore e si stabilizza più velocemente, ma ad un livello più basso. Nessuna tecnica è adatta per tutti gli scenari: l'una è più forte dove l'altra è più debole. Per le applicazioni real-time, dove l'importante è il determinismo e la prevedibilità, AOT è molto più adatto perché presenta meno variabilità nelle prestazioni.
	\begin{frame}{Fiji Virtual Machine}
\centering\includegraphics[scale=0.33]{fijiarch}
\end{frame}
	\section{Prima valutazione sull'utilizzo di Android in contesti real-time}
Fin dalla sua nascita Android ha generato moltissimo interesse intorno a sé. Il fatto di essere open-source gli permette di essere ben studiato e compreso. Inoltre chiunque può provare a fare dei miglioramenti o ad adattarlo in base alle proprie esigenze. Ricercatori e non hanno provato a fondo le funzionalità offerte, portando e proponendo modifiche a vari livelli per scopi diversi: sicurezza, uso nell'industria, ecc. In molti hanno anche studiato la possibilità di utilizzarlo in contesti real-time. 

Tuttavia, Android non è stato pensato per un utilizzo in contesti con seri vincoli temporali. Molte scelte, architetturali e non, lo penalizzano in quest'ottica. Di seguito ne verranno analizzate alcune.

\subsection{Garbage Collection}
Il garbage collector di Android è di tipo stopping-the-world, e non può essere eseguito concorrentemente con l'applicazione. Inoltre, ogni applicazione in esecuzione ha un suo garbage collector. Il runtime controlla lo stato di tutti i thread, e il garbage collector viene eseguito solo quando nessun thread legato ad un processo è in esecuzione. Questo significa che l'applicazione è ferma mentre il runtime pulisce la memoria. La strategia di pulizia è di tipo mark-sweep, con tutti i vantaggi e gli svantaggi discussi nella Sezione~ref{sec:gc}.

\subsection{Scheduler}
Android utilizza lo stesso algoritmo di scheduling del kernel Linux, ovvero \textbf{Completely Fair Scheduling} (CFS), un algoritmo basato sul concetto di virtual clock. Quest'ultimo misura la quantità di tempo di processore che, in un sistema completamente fair, sarebbe stata data ad un processo in attesa. Linux non memorizza questa informazione, ma la calcola a partire da una struttura simile alla seguente:
\begin{lstlisting}[language=c, caption={Entità schedulabile in Linux}, label={lst:schedentity}]
struct sched_entity {
	...
	u64 exec_start;
	u64 sum_exec_runtime;
	u64 vruntime;
	u64 prev_sum_exec_runtime;
	...
}
\end{lstlisting}
Quando un processo è assegnato ad una CPU, \texttt{exec\_start} è aggiornato all'istante attuale e il tempo di esecuzione è memorizzato in \texttt{sum\_exec\_runtime}. Quando il processo lascia la CPU \texttt{sum\_exec\_runtime} viene copiato in \\\texttt{prev\_sum\_exec\_runtime}. \texttt{sum\_exec\_runtime} è calcolato incrementalmente, cioè cresce monotonicamente. Infine, \texttt{vruntime} memorizza l'ammontare di tempo che è trascorso nel virtual clock durante l'esecuzione del processo. Quest'ultimo è incrementato della seguente quantità:
\[ delta\_exec\_weighted = delta\_exec * \frac{NICE\_0\_LOAD}{load.weight}; \]
dove \texttt{delta\_exec} è il tempo di CPU del processo e \texttt{load.weight} è il peso del processo. L'utilizzo a denominatore del peso può essere considerato come un fattore di correzione. Task con alta priorità (e con basso valore nice) avranno peso maggiore. Di conseguenza l'incremento di \texttt{vruntime} sarà piccolo. Run-time virtuale e fisico sono uguali quando per task con $nice = 0$ e priorità 120, cioè quando $load.weight = NICE\_0\_LOAD$. Solitamente un aumento di nice di 1 unità risulta in un tempo di CPU minore di circa 10\%. 

La coda di esecuzione è mantenuta in un albero rosso nero e ogni coda (una per CPU) memorizza un campo \texttt{min\_vruntime}. Quest'ultimo rappresenta il più piccolo \texttt{vruntime} tra tutti i processi nella coda (di conseguenza può solo aumentare, e mai diminuire). Le chiavi per i nodi dell'albero rosso nero sono date da $vruntime - minruntime$, per ogni processo nella coda.

Quando lo scheduler è invocato, il kernel prende il task con la chiave minore (che sarà memorizzato nella posizione più a sinistra), e gli assegna la CPU. Quindi gli elementi con chiave minore sono posizionati più a sinistra, e saranno eseguiti prima.

Quando un processo esegue, il suo \texttt{vruntime} aumenta costantemente, fino a quando non si posizionerà nella parte più a destra dell'albero. Dato che questo campo aumenta più lentamente per i task ad alta priorità, loro si muoveranno verso destra più lentamente. Questo significa che loro hanno più possibilità di essere eseguiti rispetto a quelli a bassa priorità, come è giusto che sia. Se un processo è in attesa il suo \texttt{vruntime} resta inalterato, ma dato che il \texttt{min\_runtime} della cosa aumenta costantemente prima o poi q	uel processo verrà svegliato perché la sua chiave è diventata la minore. 

In questo protocollo non ci sono possibilità di starvation, dato che prima o poi tutti verranno eseguiti. Inoltre, se un task si mette in attesa per I/O verrà ricompensato con tutta la quantità di tempo che è stata necessaria per completare l'operazione. 

Lo scheduler di Linux è modulare e prevede diverse classi di scheduling per poter utilizzare diversi algoritmi/politiche per diverse occasioni. Una classe di scheduling di fatto fornisce un'interfaccia allo scheduler principale per permettere di gestire task usando diversi algoritmi. Come previsto dallo standard POSIX, Linux dispone di due classi soft real-time: \textbf{SCHED\_RR}, per politiche round robin, e \textbf{SCHED\_FIFO}, per politiche FIFO. Android però fa scheduling utilizzando prevalentemente \textbf{SCHED\_OTHER}, che non offre nessun supporto real-time.

Di conseguenza lo scheduling di Android da più importanza alla fairness, una proprietà che non interessa ai sistemi real-time. 

\subsection{Gestione di interruzioni ed eventi}
Il kernel è responsabile di notificare l'applicazione quando arriva un'interruzione o si verifica un evento. Purtroppo però nessuna componente coinvolta in questo meccanismo ha la nozione di restrizioni temporali. Inoltre in Linux le interruzioni sono task con la massima priorità. Quindi un task in esecuzione ad alta priorità (ma non massima) può essere prerilasciato dall'arrivo di una interruzione. A causa di questo grande problema il sistema non può essere considerato completamente prevedibile.

\subsection{Framework applicativo}
\subsubsection{Costrutti e API}
Tra tutti i componenti forniti da Android, \texttt{Looper} e \texttt{Handler} sono i più problematici e pervasivi. Anche se un'applicazione non li usa esplicitamente, questi vengono implicitamente utilizzati dal runtime per controllarne il flusso, in particolare le transizioni tra gli stati di una Activity. Il problema principale è che la latenza della consegna dei messaggi a questi componenti non è prevedibile: thread con bassa priorità possono impedire a thread con priorità più alta di eseguire, senza motivo. 

\begin{figure}[h]
	\centering
	\includegraphics[width=0.7\linewidth]{looperHandler}
	\caption{Utilizzo di \texttt{Looper} e \texttt{Handler}}
	\label{fig:looperhandler}
\end{figure}
Figura~\ref{fig:looperhandler} mostra il funzionamento dei due costrutti. \texttt{Looper} si occupa di mantenere in un thread una coda di messaggi e di inviarli all'\texttt{Handler} opportuno, che li processerà. Il programmatore fornisce la logica per processare il messaggio implementando il metodo \texttt{handleMessage()} di \texttt{Handler}. Un'istanza di \texttt{Handler} è condivisa tra due thread per inviare e ricevere messaggi. Questo meccanismo è problematico quando più thread a priorità diverse inviano messaggi contemporaneamente. Ci sono due modi in cui i messaggi vengono processati. Di default l'ordine seguito è quello di ricezione. In aggiunta, però, un mittente può specificare un istante temporale nel quale il messaggio dovrà essere processato. In entrambi i casi però non viene considerata la priorità dei thread coinvolti. Se molti thread non real-time inviano simultaneamente messaggi allo stesso thread, insieme ad uno real-time, i messaggi di quest'ultimo saranno considerati solo dopo tutti i messaggi precedenti (Figura~\ref{fig:looperhandlerissue}). Inoltre, se altri thread non real-time inviano messaggi specificando un istante per processarli, la coda viene riordinata a run-time per fare in modo che questi messaggi vengano considerati in quel preciso istante. Questo significa che un thread real-time ad alta priorità può vedersi passare avanti un sacco di messaggi inviati da altri thread con una priorità molto più bassa.
\begin{figure}
	\centering
	\includegraphics[width=0.7\linewidth]{looperHandlerissue}
	\caption{Scenario di ritardo per un thread real-time}
	\label{fig:looperhandlerissue}
\end{figure}

\subsubsection{Servizi di sistema}
L'implementazione di servizi come \texttt{SensorManager} o \texttt{AlarmManager}, utilizzati fortemente in applicazioni per il sensing dell'ambiente circostante, non tengono conto di eventuali vincoli temporali. 

\paragraph{AlarmManager.} Un'applicazione che vuole impostare un allarme invia un messaggio ad \texttt{AlarmManager}. Quando l'allarme scatta, all'istante specificato dall'applicazione, questa viene notificata e un metodo di callback, definito all'invio del primo messaggio, viene eseguito. Il problema è che non viene data nessuna garanzia sul tempo trascorso dallo scattare dell'allarme al momento in cui l'applicazione ne viene a conoscenza. 

\paragraph{SensorManager.}
Un'applicazione può ascoltare l'ambiente circostante attraverso le API di \texttt{SensorManager} e fornire delle callback. Queste vengono chiamate ogni volta che un evento di interesse si verifica. Ci sono due principali problemi: 
\begin{itemize}
	\item non c'è nessun supporto per le priorità dei thread, dato che tutti gli eventi finiscono nella stessa coda. Un thread ad alta priorità può dover aspettare un sacco di thread a priorità minore prima di ricevere i dati di cui ha bisogno;
	\item la consegna degli eventi avviene tramite uno scambio di messaggi tra tutte le classi coinvolte nel sensing. Android non fornisce nessuna garanzia sul tempo necessario a consegnare questi messaggi.
\end{itemize}
	\section{Panoramica di possibili diverse architetture per Android Real-Time}
\subsection{Proposte ad alto livello}
\begin{figure}[h]
	\centering
	\includegraphics[width=0.5\linewidth]{androidPienoSupportoRT}
	\caption{Android Full real-time}
	\label{fig:androidpienosupportort}
\end{figure}
Una prima soluzione è mostrata in Figura~\ref{fig:androidpienosupportort} e considera la sostituzione di Linux con una versione real-time e l'aggiunta di una RT VM. Queste modifiche aggiungono prevedibilità e determinismo, ed è inoltre possibile aggiungere nuove politiche di scheduling attraverso le classi di scheduling e migliori strategie di gestione delle risorse. D'altra parte, però, tutti i driver utilizzati dal dispositivo devono essere implementati in un ottica real-time, e questo può portare a sforzi non necessari. La seconda modifica proposta riguarda l'aggiunta di una RT VM. Questa è considerata vantaggiosa, perché permette una gestione della memoria prevedibile. Inoltre, a seconda dell'algoritmo utilizzato, è anche possibile ottenere uno scheduling real-time, migliori meccanismi di sincronizzazione ed evitare l'inversione di priorità. Queste aggiunte sono assolutamente necessarie se si vuole ottenere una VM deterministica e prevedibile. Quest'ultima interagisce direttamente con il kernel per funzionalità come scheduling e gestione dei limiti di memoria. Non è necessario tenere il passo con le versioni di Android, perché è presente anche la DVM, ma è necessario implementare, la prima volta, una nuova VM da zero e l'interprete Dalvik. Inoltre l'interazione di due VM può essere problematica, e può essere necessario pensare a nuovi algoritmi per ottimizzare lo scheduling.

\begin{figure}[h]
	\centering
	\includegraphics[width=0.5\linewidth]{androidEstesoRT}
	\caption{Android esteso con funzionalità real-time}
	\label{fig:androidestesort}
\end{figure}
La seconda soluzione è quella di sostituire Linux una versione real-time e di aggiungere funzionalità real-time a DVM (Figura~\ref{fig:androidestesort}). I vantaggi e gli svantaggi della sostituzione di Linux sono uguali a prima. Ora però la DVm viene estesa per supportare RTSJ. Questo permette di aggiungere alla DVM tutte le caratteristiche real-time previste dalla RTSJ, come GC real-time e gestione asincrona di eventi. In questo caso è però necessario stare al passo con i rilasci di nuove versioni della VM, per poter portare le modifiche a tutti i dispositivi Android.

\begin{figure}[h]
	\centering
	\includegraphics[width=0.5\linewidth]{rtandroidParziale}
	\caption{Android con support real-time parziale}
	\label{fig:rtandroidparziale}
\end{figure}
Il terzo approccio (Figura~\ref{fig:rtandroidparziale}) si basa ancora su Linux real-time ed esegue applicazioni real-time direttamente sopra il sistema operativo, utilizzando le librerie native. Questo è un vantaggio per quelle applicazioni che non necessitano della VM. Al contrario, però, le applicazioni che hanno bisogno della VM non possono avere supporto real-time.

\begin{figure}[h]
	\centering
	\includegraphics[width=0.5\linewidth]{androidConRTHypervisor}
	\caption{Android con real-time hypervisor}
	\label{fig:androidconrthypervisor}
\end{figure}
Il quarto approccio (Figura~\ref*{fig:androidconrthypervisor}) utilizza un real-time hypervisor in grado di eseguire parallelamente Android e applicazioni real-time. Questa soluzione è simile a quella utilizzata da alcuni sistemi operativi real-time, come RTLinux, e consiste nell'eseguire task real-time parallelamente (ma con priorità maggiore) a task del kernel. Lo svantaggio è che le applicazioni real-time godono delle sole funzionalità offerte dall'hypervisor, e quindi non possono utilizzare né i servizi della DVM né quelli di Linux. Inoltre, se un'applicazione real-time si blocca, l'intero sistema potrebbe bloccarsi.

\subsection{Liberare manualmente la memoria}
Nella Sezione~\ref{sec:gcandroid} sono stati spiegate alcune criticità della GC in Android. Tuttavia, disabilitarla completamente non è una strada percorribile. Ogni processo ha la sua area di memoria e il GC viene invocato quando non c'è posto per soddisfare una richiesta di allocazione. Non liberare la memoria in questo contesto porterebbe sicuramente a comportamenti non prevedibili e distruttivi. Una soluzione più promettente potrebbe essere quella di liberare manualmente la memoria. In questo modo la probabilità che il GC sia invocato (e di conseguenza che l'applicazione venga bloccata) è minore. 

\begin{figure}[h]
	\centering
	\includegraphics[width=0.7\linewidth]{../images/androidMemoryManagement}
	\caption{Gestione della memoria in Android}
	\label{fig:androidmemorymanagement}
\end{figure}

In Figura~\ref{fig:androidmemorymanagement} è mostrata la gestione della memoria in Android. Le transizioni rappresentano eventi di sistema, come chiamate di metodi, e mostrano i file sorgente coinvolti. I rettangoli arrotondati indicano entità astratte e componenti di sistema. Le linee tratteggiate rappresentano le estensioni proposte in questa soluzione. 

Durante la compilazione di un'applicazione Android, ogni istanziazione (\texttt{new}) viene tradotta in una chiamata a \texttt{dvmAllocObject()}. Questo metodo prende come argomento un riferimento alla classe richiesta. Tra le altre cose, questo riferimento contiene la dimensione dell'oggetto che si vuole creare. La dimensione viene passata a \texttt{dvmMalloc()}, che alloca la quantità di memoria desiderata. In generale, quest'ultima è responsabile della gestione degli errori, mentre l'allocazione vera e propria viene fatta da \texttt{dvmHeapSourceAlloc()}, che ritorna il risultato senza nessuna validazione. Se non ci sono errori, il blocco viene poi convertito nell'oggetto e l'esecuzione continua. La chiamata può però fallire, ritornando un puntatore nullo e indicando la situazione della memoria. A questo punto \texttt{dvmMalloc()} prova a risolvere il problema avviando il GC, dopodiché l'allocazione viene ripetuta o viene lanciata un'eccezione (\texttt{OutOfMemoryException}). Gli oggetti che non sono marcati dopo la passata del GC sono eliminabili. L'eliminazione viene fatta da \texttt{dvmHeapSweepUnmarkedObjects()}, che rilascia la memoria chiamando \texttt{dvmHeapSourceFree()} per ogni riferimento. Android offre anche la possibilità di richiedere esplicitamente la passata del GC, chiamando \texttt{Runtime.gc()} manualmente. Le chiamate esplicite al GC hanno gli stessi effetti negativi di quelle implicite. L'idea è quindi quella di pulire la memoria manualmente, oggetto per oggetto senza chiamare il GC. Sappiamo che \texttt{dvmHeapSourceFree()} offre la possibilità di eliminare un oggetto, dato il suo riferimento. Viene quindi aggiunto alla DVM un nuovo metodo, \texttt{dvmHeapSweepSpecificObject()}, che chiama \texttt{dvmHeapSourceFree()}. In più la classe \texttt{Runtime} viene estesa con il metodo \texttt{freeObject()}, per rendere disponibile la funzionalità per le applicazioni Android. \texttt{freeObject()} riceve come argomento un oggetto da rimuovere. Calcola il puntatore al blocco che contiene l'oggetto e lo passa a \texttt{dvmHeapSweepSpecificObject()}, che lo rimuove attraverso \texttt{dvmHeapSourceFree()}. Dopodiché la memoria è disponibile per un'altra allocazione. Uno svantaggio è che \texttt{freeObject()} può essere usato solo per quegli oggetti di cui lo sviluppatore è al corrente, ma la memoria può anche essere riempita di oggetti temporanei. Una soluzione è quella di aggiungere a \texttt{Runtime} anche il metodo \texttt{verboseAllocations()}, che permette di avere dei log su tutti gli oggetti che vengono allocati durante la vita di un processo. Questo log può poi essere usato per rendersi conto degli oggetti creati e liberare la memoria di conseguenza. 

\subsubsection{Valutazione}
In Figura~\ref{fig:valutazionegestionemanualememoria} viene mostrato un confronto fra la gestione della memoria manuale e con GC. Si nota che, raggiunti più o meno i 2900kB di memoria riferita, la DVM invoca il GC e l'applicazione viene, conseguentemente, bloccata. Al contrario, con la gestione manuale, il GC non viene mai invocato, perché la memoria riferita resta sempre sotto la soglia.
\begin{figure}[h]
	\centering
	\includegraphics[width=0.7\linewidth]{valutazioneGestioneManualeMemoria}
	\caption{Valutazione della soluzione}
	\label{fig:valutazionegestionemanualememoria}
\end{figure}

Lo svantaggio e il limite evidente di questo approccio è che richiede che i programmatori gestiscano manualmente la memoria, proprio come in C o C++. Questo processo è estremamente difficoltoso, e può portare a molti leak e a gestioni scorrette. Sebbene prevenga l'invocazione del GC, gli errori introdotti dalla gestione manuale potrebbero portare l'applicazione ad un fallimento.

\subsection{Isolare una CPU per task real-time}
\begin{figure}[h]
	\centering
	\includegraphics[width=0.7\linewidth]{cpuisolation}
	\caption{Isolare una CPU per eseguire applicazioni real-time}
	\label{fig:cpuisolation}
\end{figure}
Figura~\ref{fig:cpuisolation} mostra un altro possibile approccio. In questo caso una CPU viene isolata e riservata per l'esecuzione di applicazioni soft real-time. Nessun cambiamento al codice del kernel è richiesto (anche se la versione deve essere $>=$ di 2.6) e la soluzione è portabile su tutti i dispositivi.

Dalla versione 2.6 del kernel Linux, infatti, il flag \texttt{CONFIG\_PREEMPT} rende la maggior parte del kernel prerilasciabile. Alcuni dispositivi, tuttavia, possono generare interruzioni che verranno eseguite alla priorità più alta, portando a latenze anche molto elevate dei task real-time. Il flag, da solo, non è quindi abbastanza per fornire garanzie sui tempi di esecuzione. Sfruttando l'architettura multi-core della maggior parte dei dispositivi Android, è possibile isolare una CPU da scheduler e interruzioni ed eseguire tutti i processi con requisiti real-time su di essa. 

\subsubsection{Isolare la CPU}

\paragraph{Isolcpus} \mbox{} \\
Dalla versione 2.6.9, Linux include un parametro di boot, \textit{isolcpus}, che permette di stabilire una lista di processori isolati. Questi non saranno mai utilizzati dallo scheduler, eccetto che per eseguire thread del kernel e interruzioni. Un task può essere eseguito su una CPU isolata utilizzando la chiamata di sistema \texttt{sched\_affinity} o il comando \texttt{taskset}.

Isolcpus è un parametro di boot, e quindi non può essere cambiato dopo l'avvio. Settare questo parametro in Android è complicato, perché molti produttori utilizzano dei bootloader proprietari. È dunque necessario fare il flash dell'immagine di Android modificata per utilizzare questo flag.

\paragraph{Cpuset} \mbox{} \\
Mentre Android esegue un'applicazione, i servizi in background e gli altri programmi possono rallentare l'esecuzione. Per risolvere questo problema viene utilizzata una funzionalità del kernel chiamata \textit{cgroups} (control groups) che limita l'utilizzo di CPU per un insieme di processi.

Ci sono due meccanismi usati da Android per influenzare lo scheduling: il \textit{nice} level e i cgroups. Il nice level influenza lo scheduling fair, nel senso che maggiore è il nice level meno frequentemente il processo verrà eseguito. Apparentemente questo potrebbe assicurare che l'applicazione in foreground non sia influenzata da chi è in background, ma, in pratica, non basta, perché molte applicazioni e servizi possono essere in background e avere services in attesa di esecuzione. Quindi Android usa anche i cgroups per distinguere tra foreground e background, e sono attivi per default nel kernel. Inoltre, esiste una funzionalità chiamata cpuset che assegna singole CPU agli cgroups. Lo scopo è quello di porre restrizioni sulle risorse utilizzabili da un processo. Android, però, non l'ha abilitato per default, e deve essere fatto a mano. I cpusets sono rappresentati come directory in un file system pseudogerarchico, dove la top directory (\texttt{/dev/cpuset}) rappresenta l'intero sistema e ogni cpuset che è figlio di un altro cpuset contiene un sottoinsieme delle CPU e dei nodi di memoria utilizzabili dal padre. 

Un processo esegue solo nelle CPU corrispondenti al suo cpuset. Di conseguenza è possibile creare un cpuset per i task real-time, ed usare tutte le altre CPu per gli altri. L'idea è mostrata in Figura~\ref{fig:cpuset}.
\begin{figure}[h]
	\centering
	\includegraphics[width=0.5\linewidth]{cpuset}
	\caption{Due cpusets per isolare task real-time}
	\label{fig:cpuset}
\end{figure}

\paragraph{Vincitore} \mbox{} \\
Entrambi i meccanismi possono essere usati per isolare un core in un'architettura multi-core. Ci sono però degli svantaggi nell'uso di Isolcpus. Primo, può essere avviato solo durante il boot, spesso dovendo fare una rebiuld e il flash dell'immagine. cpuset, al contrario, può essere cambiato a run-time. Secondo, isolcpus permette allo scheduler di assegnare task alle CPU isolate, e non è possibile impedire ad un'applicazione di chiamare \texttt{sched\_affinity} sulla CPU riservata. In cpuset, invece, l'allocazione del processo deve essere fatta scrivendo esplicitamente il PID in un file usato per la configurazione. Per modificarlo è necessario avere i privilegi di root.

Tuttavia, nessuno dei due approcci può controllare dove verranno eseguiti i processi del kernel, perché questi non possono essere assegnati in una specifica CPU.

\paragraph{SMP IRQ Affinity} \mbox{} \\
I componenti hardware inviano delle interruzioni a CPU specifiche per chiedere attenzione. Queste interruzioni non sono direttamente evitabili isolando le CPU. A partire dalla versione 2.4 del kernel, Linux ha introdotto la possibilità di assegnare alcune richieste di interruzione (IRQ) a specifiche CPU. Questa possibilità è nota con il nome di \textit{SMP IRQ Affinity}, e permette di controllare quali core gestiranno quali interruzioni. Per ogni IRQ c'è una directory in \texttt{/proc/irq}, che contiene il file \texttt{smp\_affinity}, dove è possibile cambiare l'affinità interruzione-CPU. Di conseguenza si può lasciare un core isolato, e far eseguire le interruzioni sulle altre CPU, anche se non tutte sono modificabili (ad esempio le \textit{inter-processor interrupts} non lo sono).

\paragraph{Utilizzo di meccanismi di isolamento delle CPU per lo scheduling real-time} \mbox{} \\
Il meccanismo di isolamento descritto sopra permette di eseguire task in un core isolato, riducendo quindi significativamente l'interferenza subita. In aggiunta a questo serve uno scheduler real-time. Come già detto, due politiche real-time sono fornite da linux (\texttt{SCHED\_RR} e \texttt{SCHED\_FIFO}). La politica \texttt{SCHED\_DEADLINE}, che implementa EDF, è stata aggiunta nella versione 3.14, ma le versioni del kernel per Android sono ancora precedenti a quella release.

La politica più usata è dunque \texttt{SCHED\_FIFO}, che implementa un algoritmo a fixed-priority con comportamento FIFO quando la priorità è la stessa. Un task continua ad eseguire finché non si sospende volontariamente, non si blocca o non viene prerilasciato da un task a più alta priorità. \texttt{SCHED\_RR} è simile, ma se la priorità è la stessa a tutti i task viene assegnato uno slice temporale per eseguire (in \texttt{SCHED\_FIFO} un task non prerilascia un altro task con la stessa priorità).

\paragraph{Aggiustamenti generali} \mbox{} \\
Alcuni accorgimenti sono necessari:
\begin{itemize}
	\item aggiustare la frequenza della CPU per evitare variazioni dei tempi di risposta. Questo vale solo per le CPU dedicate ai task real-time;
	\item molti dispositivi hanno demoni utilizzati per risparmiare energia, e spengono le CPU non utilizzate. La CPU dei real-time deve sempre essere attiva, quindi questi demoni devono essere bloccati e le CPU isolate devono essere sempre attive;
	\item l'opzione \texttt{CONFIG\_CPUSETS} deve essere abilitata per usare cpuset. Di default non è attiva;
	\item è necessario anche attivare il flag \texttt{CONFIG\_PREEMPT}.
\end{itemize}

\subsubsection{Valutazione}
I test eseguiti confermano che la migliore configurazione è quella con isolamento utilizzando cpuset, con politica di scheduling \texttt{SCHED\_FIFO} e SMP IRQ Affinity. Utilizzando cpuset, infatti, c'è la certezza che nessun altro task venga assegnato alla CPU isolate, riducendo quindi l'interferenza. Inoltre, in 1 secondo di run-time, il jitter con l'isolamento è di appena 46 $\mu s$, mentre senza isolamento arriva anche a 1.5s. Il tempo di risposta è quindi notevolmente migliore.

Sebbene questa soluzione sia molto versatile, perché non richiede pesanti modifiche al kernel, non supporta l'interazione tra applicazioni real-time, e queste ultime non possono essere scritte in Java. Infatti, dato che non viene fatta nessuna modifica alla VM, questa continuerà a comportarsi in modo non real-time. Inoltre una CPU deve sempre rimanere attiva per eseguire i task real-time, senza possibilità di risparmiare energia. Per dispositivi con poche risorse, questa potrebbe essere una limitazione.

Infine, dato che non è possibile impedire l'interferenza di alcune interruzioni, questo approccio è adatto solo per i sistemi con carateristiche soft real-time, cioè quando la mancanza sporadica delle deadline non compromette il funzionamento dell'applicazione.
	\section{RTDroid}
RTDroid si pone l'obiettivo di aggiungere supporto real-time ad Android nella sua interezza. Questo significa che tutti i problemi sopra riportati devono essere corretti in modo da fornire un supporto completo e che dia garanzie solide. In questa estensione, solamente un processo di livello utente (l'applicazione) è in esecuzione. 

Per risolvere tutti i problemi di Android è necessario un redesign profondo che arriva fino al livello del kernel. Quest'ultimo viene sostituito con un kernel real-rime, LinuxRT o, meglio, RTEMS. Anche la VM di Android viene sostituita, con Fiji. Sopra questo nuovo strato vengono offerte le ''classiche'' API Android, con l'aggiunta di una serie di estensioni che correggono specifici problemi riscontrati. Queste API aggiuntive rispettano la RTSJ. 

\subsection{Looper e Handler}
Ad ogni messaggio viene assegnata una priorità, attraverso due modalità:
\begin{itemize}
	\item\textit{inheritance}: il messaggio eredita la priorità del mittente;
	\item\textit{inheritance + specified}: il mittente può specificare una priorità relativa a tutti i messaggi inviati.
\end{itemize}
Dopodiché viene definita una coda per ogni priorità, con associata un \texttt{Looper} e un \texttt{Handler}. Messaggi a priorità più bassa non ritarderanno quelli a priorità più alta. Per avere garanzie sulla quantità di memoria utilizzata, le code possono essere dimensionate staticamente.

\subsection{AlarmManager}
Sia la registrazione che la consegna devono essere ridefinite per un completo supporto real-time. Per la registrazione vengono utilizzati degli alberi rosso neri (Figura~\ref{fig:rtalarm}), così da rendere il processo prevedibile sulla base della complessità delle operazioni sull'albero. 
\begin{figure}[h]
	\centering
	\includegraphics[width=0.7\linewidth]{rtalarm}
	\caption{Registrazione di allarmi real-time}
	\label{fig:rtalarm}
\end{figure}

L'albero principale mantiene i timestamp delle registrazioni e dei puntatori ad altri alberi, che ordinano i timestamp sulla base della priorità del richiedente. Di conseguenza una registrazione include solamente due inserimenti. Organizzando gli alberi in base alla priorità si ha la garanzia che un messaggio diretto ad un thread a bassa priorità non ritardi uno a priorità più alta. In questo modo, anche se un thread a bassa priorità registra un sacco di eventi (più di quanti ne possono essere gestiti), i thread a priorità più alta non verranno in nessun modo toccati.

Per la consegna viene definito un \texttt{AlarmManager} thread a cui è assegnata la priorità più alta di tutte. Questo thread sostituisce il classico meccanismo di invio di messaggi di Android. Si sveglia ogni volta che c'è un inserimento in un albero rosso nero e pianifica un thread all'istante temporale specificato. A quest'ultimo viene associata la callback specificata dall'applicazione.

\subsection{SensorManager}
Il sensing funziona attraverso un \textit{polling thread} che periodicamente ascolta l'ambiente circostante. Questo comunica con vari \textit{processing thread}, uno per ogni sensore, che interpretano i dati ricevuti dal polling thread.

I problemi riportati vengono risolti attraverso \textit{priority inheritance}. Quando un thread con priorità \texttt{p} registra un listener per un sensore, al thread associato a quel sensore viene assegnata la priorità \texttt{p}. Se più di un thread si associano allo stesso sensore, allora al thread del sensore è associata la priorità più alta tra tutti. Anche i thread creati per eseguire le callback hanno assegnata la priorità \texttt{p}. Il polling thread ha la priorità più alta di tutte, in modo da assicurare che i dati vengano raccolti appena possibile.

Quando un'applicazione registra un nuovo listener per un sensore viene di fatto creato un nuovo percorso di consegna dal polling thread al listener. Questo percorso è isolato ed eredita la priorità del thread che l'ha creato. 

\subsection{Sostituire componenti non real-time con componenti real-time}
Il kernel Linux di Android ha molte modifiche per renderlo adatto all'utilizzo in un ambiente con risorse limitate. Questo diventa un problema quando si cerca di sostituire alcuni componenti per migliorare il supporto real-time.

\paragraph{Bionic.} Un esempio è dato dalle librerie C native. Al posto di \texttt{glibc} Android utilizza \textit{Bionic}. Bionic è una libreria C leggera e altamente semplificata ed ottimizzata in modo da poter essere utilizzata in ambienti con risolrse limitate, in particolare CPU a bassa frequenza e poca memoria. Sfortunatamente però non aderisce alla specifica POSIX e non supporta le estensioni real-time di mutex e pthreads. Una modifica di questa libreria è necessaria per utilizzare una VM real-time, come Fiji.

\paragraph{Patch del kernel incompatibili.} Android ha introdotto molte modifiche al kernel Linux, tanto che l'applicazione automatica di patch per ottenere RTLinux è impossibile, e deve essere fatta manualmente. Anche dopo averla fatta manualmente, comunque, il kernel rimane non completamente prerilasciabile e questo comporta alti tempi di attesa possibili. 

\paragraph{Funzionalità non real-time del kernel.} Il kernel Android ha due funzionalità critiche sotto l'aspetto real-time. La prima è l'\textit{out of memory killer} (OOM), che viene avviato in condizioni di bassa memoria disponibile. Analizza tutte le pagine di memoria per verificare che il sistema sia veramente in condizioni critiche ed uccide un processo selezionato. I thread in esecuzione vengono fermati e bloccati per un periodo di tempo variabile. In un contesto dove è permessa solo l'esecuzione di un processo utente, però, OOM è inutile.


	
	
	%bibliography
	\newpage
	\nocite{*}
	\printbibliography 
	
\end{document}
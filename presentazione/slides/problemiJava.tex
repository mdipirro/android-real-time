\begin{frame}{Limitazioni di Java real-time /1}
	\begin{itemize}
		\item Scheduling
		\begin{itemize}
			\item Nessun utilizzo delle priorità
			\begin{itemize}
				\item Nessuna garanzia che un thread ``critico'' venga eseguito
			\end{itemize}
		\end{itemize}
		\item Garbage collection
		\begin{itemize}
			\item Politica Stopping the world
			\begin{itemize}
				\item L'applicazione viene messa in pausa finché il GC pulisce la memoria
				\item Il ritardo introdotto è variabile e non quantificabile (dimensione della memoria, aggressività)
			\end{itemize}
			\item Tipicamente il ritardo varia da centinaia di ms a secondi, inaccettabile per applicazioni RT
		\end{itemize}
	\end{itemize}
\end{frame}
\begin{frame}{Limitazioni di Java real-time /2}
	\begin{itemize}
		\item Compilazione Just in time
		\begin{itemize}
			\item La VM compila in codice nativo i metodi eseguiti più frequentemente
			\begin{itemize}
				\item Start up veloce, ma non si sa quando verrà fatta la compilazione
				\item La compilazione è concorrente all'applicazione e provoca un ritardo non quantificabile
			\end{itemize}
			\item Viene migliorato il best case, ma la distanza tra BCET e WCET è molto elevata
			\begin{itemize}
				\item In un'applicazione RT quest'ultima dovrebbe essere il più piccola possibile per evitare WCET troppo pessimistici, che portano ad un utilizzo reale delle CPU troppo basso
			\end{itemize}
		\end{itemize}
	\end{itemize}
\end{frame}
\begin{frame}{RTDroid - Componenti}
	\only<1>{\begin{itemize}
		\item Base: Memoria scoped dimensionata staticamente
		\begin{itemize}
			\item Il dimensionamento degli scopes è in genere complicato
			\begin{itemize}
				\item In RTDroid viene fatto basandosi sulla quantità dichiarata in \texttt{manifest.xml}
				\item Punto debole: ci si affida ai programmatori
			\end{itemize}
			\item Deve essere gestita correttamente, e per questo non è direttamente esposta, ma gestita dal framework
			\item Persistent scope
			\begin{itemize}
				\item I task vengono salvati qui
				\item La memoria scoped non viene ripulita finché ci sono thread che operano in essa
				\item In questo modo nessuno paga il ritardo della GC, perché non c'è GC finché c'è almeno un thread vivo
			\end{itemize}
			\item Release scope
			\begin{itemize}
				\item Utilizzata per i job: callback (per aperiodici o sporadici) e comportamenti periodici
				\item Pulita alla fine di ogni release, quindi il job non viene interferito dal GC
			\end{itemize}
		\end{itemize}
	\end{itemize}}
	\only<2>{\begin{itemize}
		\item \texttt{RealtimeService}
		\begin{itemize}
			\item Eseguiti in thread separati dalla UI e mappati con thread real-time
			\item Task sporadici o aperiodici
			\begin{itemize}
				\item Parametri: priorità, deadline, release time e limite di memoria
				\item Nessuna conoscenza del tempo di esecuzione
				\item Nessuna garanzia di minimum inter-arrival time per i task sporadici
			\end{itemize}
			\item \texttt{Periodictask}
			\begin{itemize}
				\item Richiesto anche il periodo
			\end{itemize}
		\end{itemize}
	\end{itemize}}
	\only<3>{\begin{itemize}
		\item \texttt{RealtimeReceiver}
		\begin{itemize}
			\item Un solo \texttt{Receiver} per componente. Gli intent vengono processati uno alla volta, ma si evitano numerose allocazioni/cancellazioni (come in Android)
			\item Intent accodati alla priorità del mittente, così messaggi a priorità alta verranno visti prima
			\item Intent processati alla priorità del ricevente, così la gestione non ritarda thread a priorità alta
		\end{itemize}
	\end{itemize}}
\end{frame}
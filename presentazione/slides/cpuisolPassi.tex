\begin{frame}{Isolamento di una CPU - Azioni}
	\only<1>{\begin{itemize}
		\item Dividere le CPU in due gruppi
		\begin{itemize}
			\item Una sola CPU viene utilizzata a scopo real-time
			\begin{itemize}
				\item In questo modo è possibile fare scheduling senza preoccuparsi dei problemi derivanti da sistemi multiprocessore
			\end{itemize}
			\item \texttt{CONFIG\_CPUSETS}
		\end{itemize}
		\item Gestire l'affinità interruzioni-CPU 
		\begin{itemize}
			\item La gestione immediata delle interruzioni (eseguita a priorità massima) viene gestita sulle CPU non dedicate al real-time
			\begin{itemize}
				\item Ma non tutte le interruzioni possono essere controllate (es. Inter-processor interruptions)
			\end{itemize}
			\item \texttt{smp\_affinity}
		\end{itemize}
	\end{itemize}}
	\only<2>{\begin{itemize}
		\item Scheduling con \texttt{SCHED\_FIFO}
		\begin{itemize}
			\item Oppure \texttt{SCHED\_RR}, ma ci sono ritardi causati dalla politica round robin
			\item \texttt{SCHED\_DEADLINE} non è supportata da Android
		\end{itemize}
		\item Rendere il kernel prerilasciabile 
		\begin{itemize}
			\item \texttt{CONFIG\_PREEMPT}
		\end{itemize}
		\item Fissare la frequenza della CPU
		\begin{itemize}
			\item Frequenze diverse possono causare differenze nei tempi di completamento 
		\end{itemize}
		\item Inibire il risparmio energetico
		\begin{itemize}
			\item Android impone che le CPU vadano in risparmio energetico quando idle
			\item Questo causa un jitter altissimo dovuto ai cambiamenti di stato
		\end{itemize}
	\end{itemize}}
\end{frame}
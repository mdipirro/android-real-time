\begin{frame}{Limitazioni di Android real-time}
	\only<1>{\begin{itemize}
		\item La memoria disponibile è generalmente poca
		\begin{itemize}
			\item Garbage collection molto aggressiva
			\item La politica resta stopping the world, e la maggiore aggressività implica che i ritardi introdotti sono maggiori
			\item Se la memoria disponibile è molto bassa, la ''coda'' dei pronti viene salvata in memoria secondaria
			\begin{itemize}
				\item Il tempo richiesto per lo scheduling diventa così altissimo
			\end{itemize}
		\end{itemize}
	\end{itemize}}
	\only<2>{\begin{itemize}
		\item Completely Fair Scheduler
		\begin{itemize}
			\item La coda dei pronti è gestita con un albero rosso nero
			\begin{itemize}
				\item Costo $O(log(n))$ per inserimenti e cancellazioni, molto maggiore di $O(1)$ richiesto da un array FIFO indicizzato per priorità
				\item Costo maggiore per la memorizzazione in memoria
			\end{itemize}
			\item CFS ha come obiettivo la fairness, in contrasto con i sistemi real-time
			\begin{itemize}
				\item Un thread ad alta priorità può essere ''scavalcato'' da uno a priorità molto più bassa
				\item Non si tiene conto delle deadlines
			\end{itemize}
			\item Due politiche real-time sono supportate dal kernel Linux di Android
			\begin{itemize}
				\item \texttt{SCHED\_FIFO}
				\item \texttt{SCHED\_RR}
			\end{itemize}
			ma di default viene utilizzata \texttt{SCHED\_OTHER}, che non tiene conto della priorità
		\end{itemize}
	\end{itemize}}
	\only<3>{\begin{itemize}
		\item Scambio di messaggi
		\begin{itemize}
			\item \texttt{Handler} e \texttt{Looper}
			\item Nessun supporto delle priorità
			\item Coda ordinata dinamicamente: un messaggio proveniente da un thread ad alta priorità non ha garanzia di essere ricevuto velocemente
		\end{itemize}
		\item Servizi di sistema
		\begin{itemize}
			\item \texttt{AlarmManager}
			\begin{itemize}
				\item Nessuna garanzia sul tempo di notifica
			\end{itemize}
			\item \texttt{SensorManager}
			\begin{itemize}
				\item Nessun supporto alle priorità: thread a priorità più alta possono essere notificati dopo thread a priorità bassa
				\item La notifica utilizza \texttt{Handler} e \texttt{Looper}, con ritardi variabili
			\end{itemize}
		\end{itemize}
	\end{itemize}}
\end{frame}
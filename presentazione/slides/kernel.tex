\begin{frame}{RTDroid - Kernel /1}
	- RT Linux
	\begin{itemize}
		\item Patch \texttt{CONFIG\_PREEMPT} manuale
		\begin{itemize}
			\item A causa delle troppe modifiche fatte da Android
		\end{itemize}
		\item Cambiamenti di Android
		\begin{itemize}
			\item Out of Memory Killer
			\begin{itemize}
				\item Analizza la memoria quando è bassa e uccide un processo attivo
				\item Momento di attivazione e ritardo introdotto non prevedibili
			\end{itemize}
			\item CPUFreq
			\begin{itemize}
				\item Adatta la frequenza delle CPU alla situazione della batteria
				\item Frequenze diverse comportano tempi di esecuzione e risposta diversi
				\item Differenze tra BCET e WCET elevate
				\item Difficile prevedere quando avverrà l'adattamento e a che frequenza
			\end{itemize}
		\end{itemize}
	\end{itemize}
\end{frame}
\begin{frame}{RTDroid - Kernel /2}
	+ RTEMS
	\begin{itemize}
		\item Event-driven
		\begin{itemize}
			\item Nessun ritardo dovuto al tick dell'orologio
			\item Nessuno spazio aggiuntivo richiesto per la coda dei ``pronti ma non visti''
		\end{itemize}
		\item Ottimizzato per dispositivi embedded
		\item Scheduling configurabile
		\begin{itemize}
			\item Fixed-priority di default
			\item 256 livelli di priorità
		\end{itemize}
		\item Gestione delle risorse con BPIP o PCP
		\item Gestione immediata delle interruzioni il più breve possibile
		\begin{itemize}
			\item Tempo di gestione dipendente dall'architettura
			\item La parte non immediata viene gestita dall'applicazione
			\item Ogni interruzione è identificata da un numero: 256 livelli disponibili
		\end{itemize}
	\end{itemize}
\end{frame}
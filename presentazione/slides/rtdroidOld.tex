\begin{frame}{RTDroid - Prime modifiche /1}
	\begin{itemize}
		\item \texttt{RT Handler}
		\begin{itemize}
			\item Messaggi con priorità
			\begin{itemize}
				\item Inheritance
				\item Inheritance + specified
			\end{itemize}
		\item In questo modo ogni messaggio ha associata la priorità del suo mittente, e il ricevente può processare prima i messaggi più importanti
		\end{itemize}
		\item \texttt{RT Looper}
		\begin{itemize}
			\item Coda FIFO con priorità
			\begin{itemize}
				\item Dimensionabile staticamente inserendo in \texttt{manifest.xml} il numero massimo di messaggi che un thread può ricevere
				\item Se la dimensione non è corretta alcuni messaggi di thread a bassa priorità potrebbero non essere recapitati
			\end{itemize}
		\end{itemize}
	\end{itemize}
\end{frame}
\begin{frame}{RTDroid - Prime modifiche /2}
	\begin{itemize}
		\item \texttt{AlarmManager}
		\begin{itemize}
			\item Ordinamento con alberi rosso neri
			\begin{itemize}
				\item Un albero ordina i timestamps degli allarmi
				\item Un albero per timestamp ordina i richiedenti per priorità
				\item Assicura che chi ha priorità più alta venga servito prima, ma potenzialmente chiede molta memoria
			\end{itemize}
		\end{itemize}
		\item \texttt{SensorManager}
		\begin{itemize}
			\item Polling thread con priorità massima
			\begin{itemize}
				\item Così i dati raw vengono raccolti nei momenti giusti, senza rischiare che un ritardo nel raccoglimento causi errori nelle risposte
			\end{itemize}
			\item Processing thread con priority inheritance
			\begin{itemize}
				\item I thread che interpretano i dati raw (uno per sensore) eseguono alla priorità massima di chi ha chiesto i dati
				\item Non preserva l'indipendenza, perché anche i thread a priorità più bassa beneficiano del boosting
			\end{itemize}
		\end{itemize}
	\end{itemize}
\end{frame}
\begin{frame}{Conclusioni - Fuchsia}
	\only<1>{
		\vspace{-0.5cm}
		\centering\includegraphics[scale=0.25]{androidtofuchsia}
	}

	\only<2>{
		\begin{itemize}
			\item Nuovo sistema operativo attualmente in fase di sviluppo
			\item Kernels real-time nativi
			\begin{itemize}
				\item Little kernel per dispositivi con risorse limitate
				\item Magenta (by Google) per dispositivi embedded con vincoli di risorse meno stringenti
				\item Scritti prevalentemente in C/C++
			\end{itemize}
		\end{itemize}
	}
	\only<3>{
		\begin{itemize}
			\item Le applicazioni sono scritte in Dart
			\begin{itemize}
				\item È presente una VM
				\item Tutti gli oggetti sono allocati sullo heap
				\item Due tipi di oggetti
				\begin{itemize}
					\item Short-lived, allocati in un'area chiamata \textit{new generation} la cui collection è ottimizzata e veloce (circa 2 $ms$)
					\item Long-lived, allocati nello heap ``classico''
					\item I primi hanno un tempo di vita molto corto e non occupano molto spazio, quindi sono adatti ad un uso in contesti real-time
				\end{itemize}
				\item Modello di concorrenza a \textit{isolates}, simili agli attori di Scala/Akka
				\begin{itemize}
					\item La comunicazione avviene tramite lo scambio di messaggi e non con memoria condivisa
					\item Facilmente analizzabile staticamente e senza rischio di data races
				\end{itemize}
			\end{itemize}
		\end{itemize}
	}
\end{frame}